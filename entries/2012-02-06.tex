%%% Research Diary - Entry
\documentclass[11pt,letterpaper]{article}

% Working date: The date this entry is describing. Not necessary the date of last edit.
\newcommand{\workingDate}{\textsc{2012 $|$ Feb $|$ 6}}

% Name and institution must preceed call to researchdiary.sty package
\newcommand{\userName}{Ben Keller}
\newcommand{\institution}{McMaster University}

% The guts and glory
\usepackage{researchdiary}

% Begin document
\begin{document}

\subsection*{To Do}
\begin{bullets}
\item[\checkmark] Figure out if the stellar gas ionization is actually affecting SFR
\item[\textleaf] Replicate figure 5 from Hopkins 2011
\item \sout{Track down cause of ChaNGa crash at time \~0.3}
\item Get to the bottom of the rung issue with the RTFORCE
\end{bullets}

\textleaf : \textit{In Progress} \qquad \checkmark : \textit{Completed}

\section*{Daily Log}
\subsection*{Tuesday}
I had a discussion with James today showing him the $\rho$ vs. $T$ delta 
diagram I produced, as well as the SFR vs. time plot for Tom's ICs run to test
the new ionization hack we added (they show that the SFR is not suppressed).  
James thinks that the hack may not be suppressing SFR because of resolution 
issues (we aren't resolving clusters of star formation).  James will send
me a higher resolution IC soon to test this hypothesis with.  He also suggested
using a simple periodic box, but I will need to learn how to set that up.
\subsection*{Wednesday}
I received an email from James yesterday asking about the RTF/RTFORCE rung 
issue.  He wants to understand the origin of the small steps.  He said that
what we need is a single output with many rungs that we can compare RTF and
RTFORCE.  James thinks that the cause of the rung issue is the result of 
evolution.  The cause could be one of the following:
\begin{enumerate}
\item Very high density (small smoothing h) $dt \approx h/c$
\item Very high temperatures (large c) $dt \approx h/c$
\item Very high accelerations $dt \approx \sqrt{\epsilon/a}$
\item Very rapid cooling $dt \approx E/\dot E$
\item Very large convergent velocities $dt \approx h/|dv|, dv < 0$
\end{enumerate}

In order to test this, James has asked me to run a param file with nSteps=0 
using an intermediate output for the IC.  This will produce of a dump of many
properties, most importantly dt for all the particles.  James wants to see
a plot of $dt$ vs $\rho$ for the following versions of Greg's ICs until the
iRung blows up, and the different executables for each of these are at (on
orca):
\begin{description}
\item[DRTFORCE] \verb!~/gasoline_ionization/gasoline.DRTFORCE!
\item[DRTF] \verb!~/gasoline_ionization/gasoline.DRTF!
\item[Neither DRTF or DRTFORCE] \verb!~/gasoline_ionization/gasoline.NORT!
\end{description}
I've stored each of these runs in a directory corresponding to the executable
suffix at \verb!/work/kellerbw/iRung_Test/! on orca.
\subsection*{Thursday}

\end{document}
