%%% Research Diary - Entry
\documentclass[11pt,letterpaper]{article}

% Working date: The date this entry is describing. Not necessary the date of last edit.
\newcommand{\workingDate}{\textsc{2012 $|$ Mar $|$ 12}}

% Name and institution must preceed call to researchdiary.sty package
\newcommand{\userName}{Ben Keller}
\newcommand{\institution}{McMaster University}

% The guts and glory
\usepackage{researchdiary}

% Begin document
\begin{document}

\subsection*{To Do}
\begin{bullets}
\item Rewrite Chris' density profile code to generate SPH ICs
\end{bullets}

\textleaf : \textit{In Progress} \qquad \checkmark : \textit{Completed}

\section*{Daily Log}
\subsection*{Monday}
I fixed the error James and I found in the tipsy version of oscar's ICs that I
generated
\subsection*{Thursday}
Tonight I am going to dig into Chris and Patrick's code to generate an SPH IC
comparable to the AMR IC Chris is using.  The parameters for his IC are:\\
\center
\begin{tabular}{| c | c | c | c |}
\hline
Variable & Units & Value & Description\\
\hline
$Z_{min}$ & kpc & $-8$ & Bottom of the simulation volume\\
$Z_{max}$ & kpc & $8$ & Top of the simulation volume\\
$X_{min}$ & kpc & $0$ & Left of the simulation volume\\
$X_{max}$ & kpc & $1$ & Right of the simulation volume\\
$Y_{min}$ & kpc & $0$ & Back of the simulation volume\\
$Y_{max}$ & kpc & $1$ & Front of the simulation volume\\
\hline
\end{tabular}
Since the glass is a cube, I simply need to stack 16 cubes on top of each other
to get the $1\times\1\times16$ kpc volume of Chris' ICs.  In Chris' original 
code, $u_2$ is the density I want to determine.  I'll simply write a python 
script that produces $u_2$.  It looks like I need another function, 
\verb!u1primed! from Chris, so I've sent him an email asking about it.
\end{document}
