%%% Research Diary - Entry
\documentclass[11pt,letterpaper]{article}

% Working date: The date this entry is describing. Not necessary the date of last edit.
\newcommand{\workingDate}{\textsc{2012 $|$ Apr $|$ 24}}

% Name and institution must preceed call to researchdiary.sty package
\newcommand{\userName}{Ben Keller}
\newcommand{\institution}{McMaster University}

% The guts and glory
\usepackage{researchdiary}

% Begin document
\begin{document}

\section*{Research Road Map}
\subsection*{Ultimate Goal}
Produce realistic (physically accurate) Starformation histories and rates in
Milky-Way type galaxy simulations.  This means that their feedback properties
and their SFR/SF histories must match observations.  These observations can
be found in Leroy et al. 2008,  Walter 2008 (the THINGS paper), Kennicutt 2008, etc.
\subsection*{How to accomplish this}
I am trying to tackle the problem of overcooling and hyperactive starformation
by incorporating more/better physics into the SPH simulation code.  These are 
either improvements on the existing subgrid physics (to better match the 
physical conditions of the galaxy) or to add new physics (some process that is
either unresolved, or is more than just gravity+hydro).  As of today, I have
three main prongs that I will be using to try and achieve my goal
\begin{description}
\item[RT Force] This is an implementation of the Ritchie and Thomas multiphase
SPH described in Ritchie \& Thomas 2001.  James has already written this into
gasoline, but I will be testing it out extensively.  The hope is that this
technique will somehow improve the SFR in galaxies because of its better 
handling of fluids made of particles with different densities
\item[Simple UV Ionization] Elizabeth Harper-Clark's Thesis describes the 
disruption of starforming regions by the clusters they form.  She found an 
empirical relation between the amount of stellar mass a Giant Molecular Cloud
needs to form in order to ionize itself and disrupt star formation in it of 
roughly 7:1 gas:star mass.  James and I banged together a really quick 
implementation of this in gasoline, and it just heats up neighbours of a newly
formed star to 10,000K and turns off their cooling for a while.  Hopefully this
will kill off starformation by obliterating clouds that form a lot of stars.
We will need decent resolution, though, so that the GMCs can actually have
a reasonable number of particles in them.
\item[Hopkins BH Accretion] Gasoline already has some support for sink 
particles.  I'm going to add the accretion model from Hopkins 2008 to gasoline,
and do some simulations of galaxies with SMBH sink particles using this better
accretion.  Hopefully this will give us some feedback (How?) that will suppress
starformation.
\end{description}

\pagebreak
\subsection*{To Do}
\begin{bullets}
\item[\checkmark] Rewrite Chris' density profile code to generate SPH ICs
\item[\checkmark] Generate Tipsy file from Chris' IC
\item Add Chris' potential to gasoline's pstGravExternal
\item Run Chris' Disk adiabatically
\item Run Chris' Disk with fabiogal parameter
\end{bullets}

\textleaf : \textit{In Progress} \qquad \checkmark : \textit{Completed}

\section*{Daily Log}
\subsection*{Wednesday}
Today I finished the first draft of the poster abstract for CASCA.  I've also 
transcribed my research road map and downloaded some background reading.
\subsection*{Friday}
Today I \textit{really} got Chris' IC particles finished.  I needed to correct
a number of bugs.  First and foremost, I forgot that the the distribution 
function will automatically try to stretch the z-dimension to its appropriate
size, so I didn't need to make a non-cubic glass to seed the thing.  I've
also made separate top and bottom halves, because the reflection way I was
doing it previously was a bad idea.  I just ``glue'' the top and bottom slices
together by catting them into a common file.  My next goal is to turn the 
ascii ICs into a tipsy file.

Finished making a tipsy file of the ICs!  The script \verb!mktipsy.py! takes
the ascii IC position and temperature values and generates initial 
conditons.  All this stuff is in \verb!Research/code/chris_IC!

I need to also implement an additional external potential into gasoline to 
capture the potential that Chris is using.  Without it, his ICs will blow up.
James suggested I start using the \verb!pkdBodyForce! in \verb!pkd.c! and 
modify it to fit with Chris' potential.  Once this is done, we will be able to
actually run the IC.  As a first test, James wants me to run it adiabatically
(no cooling or starformation).  If it is working, it should just sit happily 
and not really evolve.  Once this is working, we will want to run it using the
fabiogal parameter file I placed in \verb!Research/code/chris_IC!.  I have some
nice clear objectives for the immediate future.  Excellent.
\end{document}
