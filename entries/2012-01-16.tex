%%% Research Diary - Entry
\documentclass[11pt,letterpaper]{article}

% Working date: The date this entry is describing. Not necessary the date of last edit.
\newcommand{\workingDate}{\textsc{2012 $|$ Jan $|$ 16}}

% Name and institution must preceed call to researchdiary.sty package
\newcommand{\userName}{Ben Keller}
\newcommand{\institution}{McMaster University}

% The guts and glory
\usepackage{researchdiary}

% Begin document
\begin{document}

\subsection*{To Do}
\begin{bullets}
\item[\textleaf] Replicate figure 5 from Hopkins 2011
\item[\checkmark] Run ionization test cases
\item Track down cause of ChaNGa crash at time \~0.3
\end{bullets}

\textleaf : \textit{In Progress} \qquad \checkmark : \textit{Completed}

\section*{Daily Log}
\subsection*{Monday}
Over the weekend I received and email from Greg suggesting I remove the 
\verb!-DRTF! flag from gasoline to speed up the running (specifically the
problem with ;arge values of \verb!iRung!.)  This worked, and James suggested I
try running with the \verb!-DRTFORCE! enabled.
\subsection*{Tuesday}
I've started a test of Greg's IC with the \verb!-DRTFORCE!.  We shall see if it
is slow or not soon.  I'm also going to finish reproducing the Hopkins figure, 
using \verb!Research/code/scripts/mass_inflow.py!.  I've copied all the mass
inflow code I've been playing with to 
\verb!Research/code/scripts/mass_inflow_test.py!.  It looks like I may need to
do some more playing, as there is some issue with reproducing the values of
\verb!dM/dt! that are found in Hopkins figure 1.  I think this may be due to
the low age of the galaxy.  We'll see what happens if I evolve it further.
\subsection*{Thursday}
Today was a very productive day.  We had a group meeting early, at 10AM, in
which I showed off the pynmovies I made.  After lunch, I met with James and
we discussed the direction my work will be going in coming months, with a focus
on a potential paper extending Cameron Hummels work on suppression of star 
formation through metal cooling.  What James wants to do is produce a comparison
of three runs, all with cosmological initial conditions: one with just metal
cooling, one with metal cooling plus Rayleigh-Taylor subgrid modelling, and one
with RT forces and the ionization feedback.  James hopes this will let us 
produce a paper in just a few months.  I believe he suggested we use Greg's 
g1536 initial conditions as a starting point to run these test cases.  James
also sent me a new IC of a simple MW-type galaxy.  I've stored it at 
\verb!Research/data/isol_galaxy! on fox2, and I am running a version of it
with and without ionization.  James also wanted to take a look at starformation
rates, so I've written a tool to read and plot data from the XDR-formatted 
starlog files that gasoline produces.  This tool is at 
\verb!Research/code/scripts/starlog.py!.  I'll be extending it as time allows.
\subsection*{Friday}
James and I found (and subsequently fixed) a major bug in gasoline where it was
not actually ionizing and heating the gas around a new star, but the star 
particle itself!  I've restarted all the jobs that were affected by this with 
the fixed version of gasoline.  James also suggested I generate some plots to 
examine the results:  one of a temperature histogram (particles in T bins),
and one of $\log T$ vs  $\log\rho$ to see if the gas chunks are actually getting
ionized.  I've also organized my files on orca a bit more.  Here's where the
runs are all currently stored:
\begin{description}
\item{\verb!/work/kellerbw/g1536_noionize_DRTFORCE!}\\ Greg's g1536 ICs, run with
UV ionization disabled, and using the RTFORCE RT subgrid flag on.  Metal cooling
is used here too.
\item{\verb!/work/kellerbw/g1536_ionize_DRTFORCE!}\\ Greg's g1536 ICs, run with
UV ionization enabled, and using the RTFORCE RT subgrid flag on.  Metal cooling
is used here too.
\item{\verb!/work/kellerbw/isol_galaxy_ionization_DRTFORCE!}\\ James' simple 
galaxy IC, run with UV ionization enabled, and using the RTFORCE RT subgrid 
flag on.  Metal cooling is used here too.
\item{\verb!/work/kellerbw/isol_galaxy_noionization_DRTFORCE!}\\ James' simple 
galaxy IC, run with UV ionization disabled, and using the RTFORCE RT subgrid 
flag on.  Metal cooling is used here too.
\item{\verb!/work/kellerbw/Tom_IC_ionization_DRTFORCE!}\\ Tom's ICs, run with
UV ionization enabled, and using the RTFORCE RT subgrid flag on.  Metal cooling
is used here too.
\item{\verb!/work/kellerbw/Tom_IC_noionization_DRTFORCE!}\\ Tom's ICs, run with
UV ionization disabled, and using the RTFORCE RT subgrid flag on.  Metal cooling
is used here too.
\end{description}
I've also tarballed up some older runs that may not be of use in the future:
\begin{description}
\item{\verb!/work/kellerbw/Tom_IC_1strun.tar.gz!}\\ The first successful full run
of Tom's ICs.  This may be crap, as I ran it with cosmological cooling, rather
than metal cooling.  Also, UV background is off.
\item{\verb!/work/kellerbw/Tom_IC_1strun_moresteps.tar.gz!}\\ The same as above,
but with tipsy outputs every 1 timesteps instead of every 10.
\item{\verb!/work/kellerbw/Tom_IC_noionization_DRTF.tar.gz!}\\ The same as 
\verb!Tom_IC_noionization_DRTFORCE!, but with the \verb!-DRTF! flag instead of
\verb!-DRTFORCE!.  This flag totally shits the bed in some density edge 
situations, so I don't think we will be using it in the future.
\end{description}
\end{document}
