%%% Research Diary - Entry
\documentclass[11pt,letterpaper]{article}

% Working date: The date this entry is describing. Not necessary the date of last edit.
\newcommand{\workingDate}{\textsc{2012 $|$ Apr $|$ 30}}

% Name and institution must preceed call to researchdiary.sty package
\newcommand{\userName}{Ben Keller}
\newcommand{\institution}{McMaster University}

% The guts and glory
\usepackage{researchdiary}

% Begin document
\begin{document}
\pagebreak
\subsection*{To Do}
\begin{bullets}
\item[\checkmark] Rewrite Chris' density profile code to generate SPH ICs
\item[\checkmark] Generate Tipsy file from Chris' IC
\item Add Chris' potential to gasoline
\item Run Chris' Disk adiabatically
\item Run Chris' Disk with fabiogal parameter
\item Run violent shocktube to compare RT Forces
\end{bullets}

\textleaf : \textit{In Progress} \qquad \checkmark : \textit{Completed}

\section*{Daily Log}
\subsection*{Monday}
Chris did send me a copy of the external potential he used, which is this 
simple form:
$$ g(z) = \frac{v_c^2z}{R^2+z^2}$$
I've added this to \verb!pkd.c! as the function \verb!pkdChrisDiskForce!
which takes arguments for the disk radius $R$ and circular velocity $v_c$.
In order to enable this feature, I've added a 3 parameters into gasoline:
\begin{description}
\item[bChrisDisk] Turn on Chris' disk potential
\item[dChrisDiskVc] Circular velocity (in km/s) for Chris' disk potential
\item[dChrisDiskR] Disk radius (in kpc) for Chris' disk potential
\end{description}


The total mass of Chris's disk slice is $1.079*10^7\Msun$, so when making a SPH
IC from it, I just give each particle an equal slice of this mass.  The IC
I generated at $Research/code/chris_IC/IC_particles.std$ contains 65,536 
particles, which gives a mass of $\approx165\Msun$ per particle.
\subsection*{Tuesday}
Today I tried to get Chris's disk running adiabatically.  I created a param file
at \verb!Research/code/chris_IC/adiabatic.param! that should contain all the 
required flags to run without cooling or starformation, with periodic boundaries
along the x and y axes.  Orca is being finicky again though, and my job has been
sitting in the queue for the past 15 minutes, so this may not be very productive
today.
\subsection*{Wednesday}
I needed to add a few things to the adiabatic param file that were missing.  
First, I needed to add the gravitational softening length, defined as:
$$h=1.4(\frac{M_{particle}}{\rho_{max}})^{1/3}$$
This value was set using the \verb!dSoft! parameter in the param file.  I also 
needed to include a timestep (which is typically the sound crossing time in the 
smallest dimension).  I determined this to be $\approx10^6yr$. This was set 
using the \verb!dDelta! parameter, scaled using the \verb!dSecUnit! found in
the log file after a quick run of the parameter file.  The scaled value I used
was \verb!dDelta = 0.00002!.
\end{document}
