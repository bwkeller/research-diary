%%% Research Diary - Entry
\documentclass[11pt,letterpaper]{article}

% Working date: The date this entry is describing. Not necessary the date of last edit.
\newcommand{\workingDate}{\textsc{  2011 $|$ 10 $|$ 17 } }

% Name and institution must preceed call to researchdiary.sty package
\newcommand{\userName}{ BW Keller }
\newcommand{\institution}{ McMaster University }

% The guts and glory
\usepackage{research_diary}
\setcounter{secnumdepth}{-1}

% Begin document
\begin{document}

\section{Daily Log}

\subsection{Wednesday}

So, before I can do any real research, I need to know how these tools
work.\\James wants me to play around with an example from Tom Quinn
using ChaNGa and gasoline. Gasoline is on imp, ChaNGa is public.
However, Tom has made some significant changes to it, so I need to get a
copy of his version. I've emailed him requesting access, but no response
yet. I set up spideroak today to keep a synced up copy of my research
stuff across multiple machines. I would've preferred git or a real VCS,
but the setup time was practically nil, so learning git inside and out
will have to wait.

\subsection{Thursday}

Finally got a reply from Tom Quinn, but it looks like it will still be a
few days (weeks?) before I can get his copy of ChaNGa. I need a UW
NetID, and I have to mail or fax(!) them a long form with my driver's
license to get a login to get in his git repo. What a waste of my time.
So, I'm going to focus on working with what I have, this afternoon I
will take a look at the ChaNGa documentation, and start looking at
gasoline. These delays are maddening, I feel a bit like I am drowning in
a backlog right now. I've downloaded the git versions of ChaNGa into
\verb!Research/code/changa_git! Charm is at
\verb!Research/code/charm_git!.

Built charm using the interactive build script. Compile ran
successfully. Ran the charm++ simplearrayhello successfully. James
mentioned a Woodrow/Djezanski (sp?) galaxy creator, I should find this.
All the charm++ tests compile successfully! The charm++ megatest ran
successfully! For compiling ChaNGa, you MUST point the \verb!$CHARM_DIR!
environment variable at the top of the charm source, not the compiled
directory. It looks like ChaNGa compiled successfully! ChaNGa teststep
ran successfully. I can't run testcosmo without IDL, skipping that test.
ChaNGa testshock ran successfully. I need to get tipsy to run
testcollapse, and I needed it anyway. Tipsy installed, needed the
following packages: \verb!xserver-xorg-dev!, \verb!libxt-dev!,
\verb!libxaw7-dev!, \verb!libncurses5-dev!. Make sure they are installed
before you run \verb!./configure!. Fixed a bug in ChaNGa's
ParallelGravity.cpp preventing it from reading the director parameters
in properly: replace \verb!FILE *fp = fopen(param.achOutName, "r" );!
with \verb!FILE *fp = fopen(achFile, "r" );!. Movie creator is SLOWWWWW,
but I'm only running on one core, so it's my own damn fault. This will
take longer than I have today, I will run it at home (MORE POWER!). Once
home, I recompiled charm++ using the multicore-linux64 to stop using
TCP/IP (probably faster).\\I've set up the movie demo to run overnight,
it should be ready to view in the morning.

\subsection{Friday}

The movie finished, it looks like crap (what do you expect with the
settings used though). Will have to write a script to turn these PPM
files into a movie.

\section{ChaNGa Documentation Notes}

ChaNGa documentation wiki is at
http://librarian.phys.washington.edu/astro/index.php/Research:ChaNGa

\begin{itemize}
\item
  Tipsy file format (should find out more about this)
\item
  Simulates gravity using Barnes-Hut tree, $O(n\log n)$ for $n$
  particles
\item
  Timesteps run with leapfrog integration
\end{itemize}


\end{document}