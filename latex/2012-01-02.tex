%%% Research Diary - Entry
\documentclass[11pt,letterpaper]{article}

% Working date: The date this entry is describing. Not necessary the date of last edit.
\newcommand{\workingDate}{\textsc{  2012 $|$ 1 $|$  2 } }

% Name and institution must preceed call to researchdiary.sty package
\newcommand{\userName}{ BW Keller }
\newcommand{\institution}{ McMaster University }

% The guts and glory
\usepackage{research_diary}
\setcounter{secnumdepth}{-1}

% Begin document
\begin{document}

\section{To Do}
\begin{bullets}
\item[\checkmark] Run Tom's ICs on gasoline.
\item[\textleaf] Replicate figure 5 from Hopkins 2011
\item[\checkmark] Make some cool movies using pynbody
\item Track down cause of ChaNGa crash at time \~0.3
\end{bullets}

\textleaf : \textit{In Progress} \qquad \checkmark : \textit{Completed}
\section{Daily Log}

\subsection{Tuesday}

Today I am going to begin replicating figure 5 from the Hopkins 2011
\textit{
Analytic Model of Angular Momentum Transport} paper. I think pynbody is
the tool for this job, so I've updated it, and I'm now reading the
documentation.

Wowee, I made some really awesome plots with pynbody! This is really
handy.\\I think I'm going to use it to make some movies. I've started a
script in \verb!Research/scripts/pynmovie.py! for autogenerating
time-frame movies of evolution using pynbody's plotting features.
Tomorrow I will need to figure out what I will need to do to replicate
the figure I need. I will probably need to add something into pynbody. I
think what I will need to do is calculate a surface integral of the mass
flow through a cylidrical slice at a given radius, using some sort of
Gauss' law type formulation. I will need to learn some neat SPH stuff to
figure out how to do this! I hope I don't need to run any more of the
simulations. I should also probably go through that paper I emailed
James about yesterday with a highlighter to prep for the group meeting
Thursday.

\subsection{Wednesday}

I was thinking about how to calculate $\dot M$ today in the shower, and
I came up with this: \[\dot M = \oint \rho \vec v \cdot d\vec S\] Where
$\rho$ is the surface density of stars + gas. I'm not sure if there will
be SPH related issues in calculating this, but I'm going to get to work
today to try and find if pynbody already has the required features to
calculate that, or if I need to add them myself.

I met with James after class, and he confirmed my worries, calculating
$\int\rho d\vec S$ will be\ldots{} troublesome. As luck would have it,
though, he and Rory have already come up with an algorithm that nicely
deals with the problem of discrete SPH particles nicely. I've emailed
Rory to ask about it. We talked about the Rahimi paper I sent him
yesterday, and I've got a better idea as to what I will be working on in
the future. I need to talk to Chris about using his results to model
supernova feedback in my simulations, so I will send him an email or
pull him aside after the next group meeting.

\subsection{Thursday}

I received an email from Rory earlier this morning, including his
formulation for calculating $\dot M$. It's contained within a python
script on his machine, mercury. It is stored at
\verb!/home/woodsrm/python/createImages.py!. The pertinent line is as
follows:

\texttt{s.g{[}'mdot'{]} = -(s.g{[}'r'{]}**2)*s.g{[}'rho'{]}*s.g{[}'vr'{]}}

This translates to this equation (for gas, as indicated by the
\verb!s.g! variables being accessed):

\[\dot M = -r^2\rho v_r\] Naturally, this will need to be calculated for
both the gas and stars. I've started a script at
\verb!Research/code/scripts/mass_inflow.py! for this work.

\subsection{Friday}

There is a strange effect going on after calculating $\dot M$, where the
face on mass flux changes sign in what looks like opposite quarters of
the galaxy. I'm going to talk to James about what is up with it.

The meeting with James was extremely productive. We derived a more
theoretically justified method for calculating the mass inflow through a
shell, that will work better for the discrete case. It is defined as
follows:

\begin{itemize}
\item
  A spherical shell with inner radius $r$ and thickness $\Delta r$
\item
  Particles within that shell are added to determine the mass flow
  through the shell. This is defined as follows:
  \[\Sigma m \frac{v_r}{\Delta r}\] We may eventually want to calculate
  something that uses the smoothing length to be a bit more
  sophisticated, but as of yet this should be sufficient for our
  purposes.
\end{itemize}

James and I also discussed the direction my work will be going in the
near future. With Chris' work on a high-resolution star formation model,
we hope to be able to better determine the cooling time for an area
(this determines the starformation rate as
$\dot\rho_* = c_*\rho\tau^{-1}$). The idea is to perhaps use a
multifluid model to model the cool molecular component along with the
hotter, lower density ISM. Currently, a naive method simply averages the
temperature of the box, which is problematic because the cooling
efficiency $\Lambda$ is a function of the temperature of the medium, and
peaks for intermediate temperatures. The cooling time is defined simply
as: \[\tau_{cool} = \frac{kT}{\Lambda}\]


\end{document}