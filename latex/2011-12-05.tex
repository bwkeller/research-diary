%%% Research Diary - Entry
\documentclass[11pt,letterpaper]{article}

% Working date: The date this entry is describing. Not necessary the date of last edit.
\newcommand{\workingDate}{\textsc{  2011 $|$ 12 $|$  5 } }

% Name and institution must preceed call to researchdiary.sty package
\newcommand{\userName}{ BW Keller }
\newcommand{\institution}{ McMaster University }

% The guts and glory
\usepackage{research_diary}
\setcounter{secnumdepth}{-1}

% Begin document
\begin{document}

\section{To Do}
\begin{bullets}
\item[\checkmark] See where to start with gasoline
\item[\textleaf] Run Tom's ICs on gasoline.
\item Track down cause of ChaNGa crash at time \~0.3
\item Familiarize myself with tipsy
\end{bullets}

\textleaf : \textit{In Progress} \qquad \checkmark : \textit{Completed}
\section{Daily Log}

\subsection{Monday}

I've emailed James about the strange gasoline hanging bug, as I've been
unable to determine the cause. He replied that I need to set a number of
MPI related environment variables as a possible solution to the problem.
These can be passed to the shell through the builtin \verb!export!:

\begin{verbatim}
export OMPI_MCA_btl=^mx,ofud
export OMPI_MCA_pml=^cm
export OMPI_MCA_mpi_leave_pinned=0
export OMPI_MCA_btl_openib_eager_rdma=0
\end{verbatim}

He also recommended I run on fewer cores, as gasoline typically runs
best when each processor handles \textasciitilde{}100,000 particles.
Also, SHARCNet's orca is optimized for jobs in groups of 24 cores. I
should try and run in multiples of 24, or for smaller jobs, multiples of
8.

\subsection{Tuesday}

The gasoline process is still running on orca, which is good. It seems a
bit slow though, perhaps James was low-balling me a bit on the number of
cores to run on. I will work mostly on the gasoline manual today.

\subsection*{Wednesday}

I unfortunately forgot to give gasoline enough run time to actually
complete the full run of Tom's ICs, so I had to restart it today with a
longer walltime. I also have tried to ``resume'' it from the stage it
was at this morning. I did so by changing the
\verb!Disk_Collapse_1e6.param! lines:

\begin{verbatim}
achInFile       = Disk_Collapse_1e6.std
nSteps      = 150
dExtraCoolShutoff   = 2
\end{verbatim}

to

\begin{verbatim}
achInFile       = Disk_Collapse_1e6.00070
nSteps      = 80
dExtraCoolShutoff   = 0
\end{verbatim}

The first two changes get gasoline to pick up at the intermediate output
from timestep 70, and the second change I believe is necessary because
there is ``extra cooling'' enabled within the first two timesteps of the
simulation, and I don't want it kicking in again at the 70th timestep.


\end{document}