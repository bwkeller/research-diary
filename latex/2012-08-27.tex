%%% Research Diary - Entry
\documentclass[11pt,letterpaper]{article}

% Working date: The date this entry is describing. Not necessary the date of last edit.
\newcommand{\workingDate}{\textsc{  2012 $|$ 8 $|$ 27 } }

% Name and institution must preceed call to researchdiary.sty package
\newcommand{\userName}{ BW Keller }
\newcommand{\institution}{ McMaster University }

% The guts and glory
\usepackage{research_diary}
\setcounter{secnumdepth}{-1}

% Begin document
\begin{document}

\section{To Do}
\begin{bullets}
\item[\textleaf] Run disk comparison - basic test
\item[\textleaf] Rerun MAGICC onestar with no cooling (bGasAdiabatic)
\item[\textleaf] Check on dESN in feedback for the onestar box
\item Debug James' smDtSmooth
\item[\checkmark] Get better onestar plots
\item Symposium day talk.
\item Read Behroozi paper
\item[\checkmark] Fancy zoom-in videos
\item[\textleaf] Re-run Chris IC with DRTFORCE
\item Make Isotropic onestar box.
\item Fix onestar box timesteps
\item Get phase plots of the g1536 galaxy
\end{bullets}

\textleaf : \textit{In Progress} \qquad \checkmark : \textit{Completed}
\section{Daily Log}

\subsection{Monday}

I think the runs of the onestar boxes are done. I'm going to produce the
proper plots and build a page on my website for it.

The g1536 run is going to take months. I don't think I'm going to get
much of any use there. Perhaps Rory has a completed run\ldots{}

Woah boy. Another epic 2 hour James meeting. I might be meeting with
Samantha this week to offload some of the large-scale stuff to her.

I've created ICs for Oscar's disk comparison project, and fired the low
resolution version up on orca. I'll see how it looks in the morning.
Oops, forgot to add a softening length. The readme says the resolution
is 80 kpc, so I'll use 1/4 of that to be safe.

\subsection{Tuesday}

I've finished running the low res version of oscar's disk. I'm going to
make a video of the results to see how things look.

I met with James. He suggests I look at the evolution of the 1/2 mass
radius as a function of time. He also gave me some ideas for my
Symposium day talk, along with an email. He suggests I look at the big
picture, and keep it positive. Avoid sarcasm, etc.

I've fired up the re-runs of the onestar box on iqaluk. I'll download
the results tomorrow when I get in.

\subsection{Wednesday}

\subsection{Thursday}

James and I had a skype conversation with Greg today.

\subsection{Friday}

Spent the entire day working on my presentation for symposium day. along
with a good meeting with James about it. In general, I need to pay more
attention to my audience. In other words, the narrative-heavy TED style
talks are good for outreach and public talks, but I need to focus on the
science details for academic talks. He gave me some good pointers:

\begin{itemize}
\item
  Remove content-free slides (intros, outlines, etc.)
\item
  Don't use fragment bullet points (give people time to think about the
  slide)
\item
  Less history
\item
  Don't just talk about my ``personal philosophy''
\end{itemize}

I'm going to revise the slides today, and send the newest version to
James when I'm done.


\end{document}