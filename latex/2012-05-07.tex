%%% Research Diary - Entry
\documentclass[11pt,letterpaper]{article}

% Working date: The date this entry is describing. Not necessary the date of last edit.
\newcommand{\workingDate}{\textsc{  2012 $|$ 5 $|$  7 } }

% Name and institution must preceed call to researchdiary.sty package
\newcommand{\userName}{ BW Keller }
\newcommand{\institution}{ McMaster University }

% The guts and glory
\usepackage{research_diary}
\setcounter{secnumdepth}{-1}

% Begin document
\begin{document}

\section{To Do}
\begin{bullets}
\item[\checkmark] Rewrite Chris' density profile code to generate SPH ICs
\item[\checkmark] Generate Tipsy file from Chris' IC
\item[\checkmark] Add Chris' potential to gasoline
\item[\checkmark] Run Chris' Disk adiabatically
\item Run Chris' Disk with fabiogal parameter
\item[\checkmark] Run violent shocktube to compare RT Forces
\end{bullets}

\textleaf : \textit{In Progress} \qquad \checkmark : \textit{Completed}
\section{Daily Log}

\subsection{Monday}

Internet is down right now. May not be able to get any work done. If it
doesn't come back up, I will just catch up on some papers. I moved the
things for working with Chris' Disk slice from
\texttt{Research/code/chris\_IC} to\texttt{Research/data/chris\_IC}.

The internet is
restored\texttt{I've started running (locally, on fox2) Chris' IC to see if I can make it through one full timestep before I blast it off to run on orca.  It crashed the first time because of a}assert(bInBox)\texttt{failure, which I believe is because the x and y range is \$(0,1)\$ instead of \$(-0.5,0.5\$. I have fixed this in the tipsy file, and if it solves the problem, I will add  it to}mktipsy.py\texttt{.  It worked}

Running Chris' IC on orca keeps crashing with the following assertion
violation: \texttt{pst.c:1177: \_pstRootSplit: Assertion}fmm -
pst-\textgreater{}bnd.fMin{[}dBnd{]} \textgreater{}= -1e-14*fabs(fmm)
\&\& fmm \textless{}= pst-\textgreater{}bnd.fMax{[}dBnd{]}' failed` It
failed at step 0.259014. I am going to re-run it to see if it keeps
occuring at the same spot. I'm also going to start a process on fox2 to
see if it is an MPI-related thing. If neither is fruitful, I will have
the wonderful job of finding out what this assertion actually means.

\subsection{Tuesday}

I met with James today. It looks like that failure is due to a particle
falling outside the boundaries (weird, I know. I couldn't find it using
\texttt{examine}) of the periodic box. I've added a line to
\texttt{pkd.c} at line 4153 that prints out the position of the particle
and the offending position causing the assertion failure. I'm going to
set it to run on fox2, and hopefully that will let me find the offending
particles.

I've also re-ran Rory's shocktube with \texttt{dAlphaConst=2.0} in the
parameter file. This is used to correct for a modification flag in the
artificial viscosity that is halving the value from what it should be
(1).

\subsection{Thursday}

Today was very productive. I met with James, and fixed Chris' IC. James
mentioned to me that if I use periodic boundary conditions, I shouldn't
have the z-axis non-periodic. It's all or nothing. I've updated that in
the param file. I've also found the cause of the \texttt{bInBox} and
\texttt{fmm} assertion failures. My $dt$ is \textit{way} too large: on
the order of one dynamical time. I need it to be a tenth or a hundredth
of this (the dynamical time of Chris' IC is $\approx 10 Myr$). What was
happening was that the occasional particle was traversing the entire
length of the box \textit{twice}. The periodic boundary code only
wrapped the particle position once, so it still ended up outside the
box. These smaller timesteps also run much faster, since way smaller
rungs are used.

James said in a email long ago that $10^5$ particles per CPU is the
ideal run state for gasoline. I'd like to run a rough benchmark to see
if this is true. I'm going to benchmark Chris' IC in
\texttt{Research/data/gasoline\_benchmark/}.


\end{document}