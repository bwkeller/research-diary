%%% Research Diary - Entry
\documentclass[11pt,letterpaper]{article}

% Working date: The date this entry is describing. Not necessary the date of last edit.
\newcommand{\workingDate}{\textsc{  2012 $|$ 9 $|$ 17 } }

% Name and institution must preceed call to researchdiary.sty package
\newcommand{\userName}{ BW Keller }
\newcommand{\institution}{ McMaster University }

% The guts and glory
\usepackage{research_diary}
\setcounter{secnumdepth}{-1}

% Begin document
\begin{document}

\section{To Do}
\begin{bullets}
\item[\textleaf] Run disk comparison - basic test
\item[\textleaf] Rerun MAGICC onestar with no cooling (bGasAdiabatic)
\item Give Greg debug information
\item[\checkmark] Symposium day talk.
\item Read Behroozi paper
\item[\textleaf] Re-run Chris IC with DRTFORCE
\item[\checkmark] Make Isotropic onestar box.
\item Fix onestar box timesteps
\item Get phase plots of the g1536 galaxy
\end{bullets}

\textleaf : \textit{In Progress} \qquad \checkmark : \textit{Completed}
\subsection{Daily Log}

\section{Tuesday}

I had a lunch meeting with James today. He is very keen on getting a
simple, working ESF model to start playing with. We don't need much in
the way of the underlying physical principle. Once we have this, we can
do a HUGE galaxy run ($10 \Msun$ particles).

I spent the afternoon helping James get familiar with git and merge his
copy of gasoline to the latest version from UW. I also created a syntax
highlighting file for lanyon entries, so that these journal entries will
look pretty and clean in vim.

\section{Wednesday}

I am \textbf{finally} getting back to work after a long two weeks of
summer school and the symposium day talk.

I'm going to try and get to the bottom of why our ESF isn't working as
well as MAGICC. I have a few ideas:

\begin{itemize}
\item
  Not enough energy being input \textbf{I Think this is it!}
\item
  HII regions saturating the disk, preventing new ionization \textbf{not
  the answer, the fraction of mgot/mwanted averaged for 1000 steps is 1
  to 6 decimal places}
\item
  SNe and ESF doing effectively the same thing, so I'm just changing the
  timing
\end{itemize}

In order to fully test this, I need that isotropic onestar box. I'm
going to make one with a uniform temperature of 1000K. The positions
will be random, but isotropic to give a cube of 1 code unit per side. I
should probably make the conditions inside similar to my SF threshold
(10 atom/cc).

After examining the result of the single star ionization run, it looks
like a very small amount of energy is being injected. At 10 atom/cc and
1000K, the additional energy put into the box is a mere \$ 10\^{}\{46\}
erg/Msun\$.


\end{document}