%%% Research Diary - Entry
\documentclass[11pt,letterpaper]{article}

% Working date: The date this entry is describing. Not necessary the date of last edit.
\newcommand{\workingDate}{\textsc{2012 $|$ Jan $|$ 09}}

% Name and institution must preceed call to researchdiary.sty package
\newcommand{\userName}{Ben Keller}
\newcommand{\institution}{McMaster University}

% The guts and glory
\usepackage{researchdiary}

% Begin document
\begin{document}

\subsection*{To Do}
\begin{bullets}
\item[\checkmark] Run Tom's ICs on gasoline.
\item[\textleaf] Replicate figure 5 from Hopkins 2011
\item[\checkmark] Make some cool movies using pynbody
\item[\textleaf] Run ionization test cases
\item Track down cause of ChaNGa crash at time \~0.3
\end{bullets}

\textleaf : \textit{In Progress} \qquad \checkmark : \textit{Completed}

\section*{Daily Log}
\subsection*{Monday}
I've begun working on the mass inflow calculator in earnest in 
\verb!Research/code/scripts/mass_inflow.py! using the algorithm James and I
discussed Friday.  It will calculate the mass flow through a shell with inner
radius $R$ and outer radius $R+\Delta R$
\subsection*{Wednesday}
I've finished up fiddling with pynbody and the mass inflow calculator, and I
feel sufficiently confident that I can now reproduce figure 5 from Hopkins
2011.
\\
I've just had an extremely eventful and productive meeting with James and Greg
Stinson.  We've implemented a new feature in gasoline, a copy of which can be
found at \verb!Research/code/gasoline_ionization!.  What the new code does is
that after each star formation event, we simulate the effects of UV ionization
on the surrounding gas by heating it to 8000 K, and turning off the cooling
features (as ionized HII cools inefficiently).  This has the effect of 
a de facto starformation killer, and will moderate the starformation rate in an
area not by removing gas, but by raising its temperature above the starformation
threshold of 1000 K (simply change dTempMax in the parameter file).  
This will also have an effect on SN feedback.  I need to
run the new code on an initial condition Greg will send me, and on Tom's ICs 
after the initial burst in starformation is over.  I need to plot that to find
out when it is!  I will also need to run this with the original gasoline as a 
comparison to see what the effect of the new physics is.  I've emailed Greg
about sending me the location for his initial conditions.  I will need to run
this with metal cooling.  I also need to add the \verb!-DDIFFUSION! flag to
the compile-time options.  I also need to remember to copy the 
\verb!cooltable_xdr! to the working directory of any run that uses the metal
cooling.
\subsection*{Friday}
I began to feel a bit over my head today.  Yesterday afternoon, I met with Greg,
and we found that gasoline was running his \verb!g1536! intial conditions very
slowly, and this was due to \verb!iRung! being knocked up to large ($\approx15$)
values after the completion of the first timestep.  I was unable to find the 
cause of this.  I've started a run of Greg's ICs with checkpointing fixed, and
with tipsy outputs at every timestep.  I've restarted all of the ionization 
comparison runs, and James advised me to keep code and data separate, with
code in \verb!~kellerbw! and data on \verb!/work/kellerbw!.  Once the comparison
runs are complete, I will reorganize my data on SHARCNet.
\end{document}
