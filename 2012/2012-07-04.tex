%%% Research Diary - Entry
\documentclass[11pt,letterpaper]{article}

% Working date: The date this entry is describing. Not necessary the date of last edit.
\newcommand{\workingDate}{\textsc{2012 $|$ July $|$ 04}}

% Name and institution must preceed call to researchdiary.sty package
\newcommand{\userName}{Ben Keller}
\newcommand{\institution}{McMaster University}

% The guts and glory
\usepackage{researchdiary}

% Begin document
\begin{document}

\subsection*{To Do}
\begin{bullets}
\item \checkmark Get phase diagram for ionization run at $\approx 100$ Myr
\item \checkmark Re-run EVERYTHING without divv, and with proper initial star masses
\item \textleaf Run with ionization, no SN
\item Run with 4 glommed together versions of the IC (in the x-y plane)
\end{bullets}

\textleaf : \textit{In Progress} \qquad \checkmark : \textit{Completed}

\section*{Daily Log}
\subsection*{Wednesday}
Met with James today.  We discussed many things.  Apparently, it is accepted 
practice (read tradition) to build stars in the following way:  2 stars are 
made from from each gas particle, each with a mass of 0.3 the initial gas 
particle, leaving 0.4 of the gas remaining.  Feedback and accretion happens, 
and gas particles below 0.1 get deleted (I assume that means they get accreted
on to the star particles?)

\subsection*{Thursday}
In looking at the phase diagram comparing the ionizing and non-ionizing 
feedback, it appears that particles are being brought up to 4000K rather than 
8000K.  I will see why this is happening.  I think this is actually reasonable.
The ionization feedback first knocks particles up to $T_{ionize}=8000$, but 
they will cool quickly to below $T_{ionizeMin}=4000$, where they will be kicked
back up again.

\end{document}
