%%% Research Diary - Entry
\documentclass[11pt,letterpaper]{article}

% Working date: The date this entry is describing. Not necessary the date of last edit.
\newcommand{\workingDate}{\textsc{2012 $|$ May $|$ 14}}

% Name and institution must preceed call to researchdiary.sty package
\newcommand{\userName}{Ben Keller}
\newcommand{\institution}{McMaster University}

% The guts and glory
\usepackage{researchdiary}

% Begin document
\begin{document}
\pagebreak
\subsection*{To Do}
\begin{bullets}
\item[\checkmark] Add Chris' potential to gasoline
\item[\checkmark] Run Chris' Disk adiabatically
\item Run Chris' Disk with fabiogal parameter
\item CASCA Poster
\end{bullets}

\textleaf : \textit{In Progress} \qquad \checkmark : \textit{Completed}

\section*{Daily Log}
\subsection*{Monday}
The benchmarking process produced some nice results (and with Amdahl's law, I
can estimate the fraction of gasoline that is parallel).  Amdahl's law states
that the estimated parallelization $P_{estimated}$ can be found from the speedup
$S$ due to using $N$ processors:
$$P_{estimated} = \frac{S^{-1}-1}{N^{-1}-1}$$
Since the speedup with 2 cores was 1.65, gasoline is $\approx 0.78$ parallel. 
If I am purely concerned with speed, I can use Gustafson's Law:
$$S(N) = N -(1-P_{estimated})(N-1)$$

But all that is a bit more complicated than I have time to deal with at the moment, so I am going to use a rough rule of thumb:  the speedup started to 
seriously diminish at about 4 cores, so I think if I just want raw POWAH, I 
will run with 1 core per 16,000 particles when running just in pthread mode.
Obviously, with MPI and SHARCnet stuff, the speedup will be different.  This is
just for running it here on angband or on fox2.

\subsection*{Tuesday}
It looks like Chris' IC is not static with the adiabatic parameters I have.  
I'm going to look into this today.  It looks like the initial density is good,
(calculated using $\rho = 1.4 \frac{m}{h^3}$).  The slice collapses pretty 
rapidly, even after the first timestep.  Perhaps gravity is too aggressive?  It
is also heating up like crazy in the central regions (adiabatic compression). I 
am moving the results of this first attempt to 
\verb!Research/data/chris_IC/adiabatic_run1!.  I'm going to try running with
the gravitational smoothing length set to the highest/lowest smoothing length
derived by pynbody. Setting it to the max value found (450 pc) drops the central
density quite a bit, but still it collapses.  The minimum value (15 pc) doesn't
do much to help either. I'm going to try setting each particle's smoothing length using pynbody.  I am an idiot.  I've given these particles \textit{half} the 
mass that they should have: I only calculated the the column density for 
positive z values. The total mass needs to be $2.152*10^7\Msun$, and the 
individual particles must have a mass of $328\Msun$.  Derp.  I'm going to try
using the average gravitational smoothing length ($30 pc$).

That didn't seem to work, so I'm pretty sure the problem is in the potential I
wrote into gasoline.  I'm going to try running once without the potential 
(it should blow up) just to be certain that I didn't really screw up the IC. 
Well, shit.  Even with the potential removed, the thing still collapses on 
itself.  I have seriously messed up creating the IC I guess.  Even turning 
gravity OFF, the IC still contracts.  The heating is coming from gravitational
potential being released (cool!).  This still makes no sense.  I wonder if it 
is because I don't have enough particles?

Well, I found out the problem.  I used \verb!dzPeriod = 8! instead of 
\verb!dzPeriod = 16!.  Fuck me.

I'm now running it with the correct periodic boundary conditions.  Weirdly
enough, it now seems to be shifting to the left (negative z).  By timestep 10, 
the peak has been shifted down to $z\approx 400$ pc.  This seems to suggest a
mistake in the gravitational potential. I'm going to re-run it without the 
external potential, and see if that changes anything.  Well, it still seems to
be drifting to the right.
\subsection*{Wednesday}
I've got too many variables to control for right now.  I'm going to start by
checking that my gravitational potential is correct.  I'll create a test 
case with a singe DM test particle, and see if it is given the appropriate 
velocity.  The potential is clearly not working.  In fact, it isn't even being
called!  I found the problem.  In master.c, there is a line checking for the
various parameters calling for external potentials, and I forgot to add the new
parametar for Chris' disk there! Derp.  It is fixed now.  I had also messed up
the unit conversion, but all is well.  I am now fairly confident that the 
potential is correct.  But still it blows up to a billion degrees in the halo!

Holy shit. After 2 timesteps, the thing is getting up to a trillion degrees.  
I am pretty sure the problem is that the initial density is way too high in
the core.  I must have fucked up making the IC.

I need to tackle this in a systematic fashion.  I'm going to start with the
barest minimum: No gas, no gravity, no external potential.
\begin{description}
\item[No gas, no gravity, no potential] Density, Temperature profiles constant (hooray!)
\item[Just gas, no gravity, no potential] Density, profile constant (actually 
gets closer to the analytic profile, temperature profile roughly constant 
(the variance increases)
\item[Gas, no self gravity, potential] Same as with no potential (roughly)
\item[Gas, self gravity, no potential] CRAZY EXPLOSION of temperature (100 
billion degree halo particles
\end{description}
\subsection*{Thursday}
I met with Rory today, and we discussed the merits of code hosting on github 
versus google code.  I think the additional collaboration tools on github make 
it a slightly more attractive option, so we will probably end up going with 
that.  I also discussed the problems I have been having with Chris' initial
condition run, and we both agree that something is up with the gravitational
potential due to the gas mass.  I'm going to try dropping the gas mass by 2 
orders of magnitude, and seeing if that reduces the temperature explosion. He 
also suggested running with a really small \verb!dDelta!.  I may try that as 
well.

Aha.  When I cut the particle mass by a factor of 100, much fewer particles
get hotter.  I am going to try re-creating the ICs using an inverted column 
density function.  Well, that didn't work.  Still too dense in the core regions.

I've started a high time resolution run, and it looks like what is happening is 
a thermal runaway around a few points at $\approx\pm3 kpc$ that then heats the
surrounding material. This is exceptionally weird.
\end{document}
