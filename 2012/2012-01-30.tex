%%% Research Diary - Entry
\documentclass[11pt,letterpaper]{article}

% Working date: The date this entry is describing. Not necessary the date of last edit.
\newcommand{\workingDate}{\textsc{2012 $|$ Jan $|$30}}

% Name and institution must preceed call to researchdiary.sty package
\newcommand{\userName}{Ben Keller}
\newcommand{\institution}{McMaster University}

% The guts and glory
\usepackage{researchdiary}

% Begin document
\begin{document}

\subsection*{To Do}
\begin{bullets}
\item Figure out if the stellar gas ionization is actually affecting SFR
\item[\textleaf] Replicate figure 5 from Hopkins 2011
\item Track down cause of ChaNGa crash at time \~0.3
\end{bullets}

\textleaf : \textit{In Progress} \qquad \checkmark : \textit{Completed}

\section*{Daily Log}
\subsection*{Thursday}
I'm a bit baffled.  The run of Tom's ICs using the DRTFORCE flag and the 
ionization (comparing the ionization on/off) doesn't seem to be producing a whit
of difference in the actual output.  As 
\verb!Research/code/scripts/compareSFR_Tom_IC_DRTFORCE.py! shows, the total
SFR is barely different, aside from the initial peak happening earlier (!).  
Also alarming is the total mass, shown by 
\verb!Research/code/scripts/compareStarMass_Tom_IC_DRTFORCE.py!.  I'm going to
write a script to try and look at the temperature vs. density as James 
suggested.\\

I've plotted and pondered the phase diagram of the gas particles in Tom's ICs
(general $T$ vs $\rho$ plotting script at 
\verb!Research/code/scripts/plot_T_rho.py!, comparison script at \\
\verb!Research/code/scripts/compare_T_rho_Tom_IC_DRTFORCE.py!).  Kevin and I
hypothesize the lack of SF quenching is due to sloppy conditions in gasoline's
ionization subroutine.  What appears to be happening is the ``heating'' of 
dense molecular gas is occuring along with (in reality cooling) of hot, thin
halo gas.  Looking at the code, I don't think this is actually what is occuring.
It should be checking that the temperature of the particles ionized is below
\verb!dIonizeTMin! and \verb!dIonizeT!.  I wonder if it has to do with the 
following lines not occuring within the if statement checking that the
temperature is below \verb!dIonizeT!:\\
\verb!      mgot += q->fMass;!\\
\verb!      ngot++;!\\
\verb!      tCoolAgain = smf->dIonizeTime+smf->dTime;!\\
\verb!      if (tCoolAgain > q->fTimeCoolIsOffUntil) q->fTimeCoolIsOffUntil=tCoolAgain;!\\
I've created an alternate gasoline source tree at 
\verb!Research/code/gasoline_ionization_bettercheck!.  In there, I've
switched the following lines in \verb!smoothfcn.c! from:\\\\
\verb!      mgot += q->fMass;!\\
\verb!      ngot++;!\\
\verb!      tCoolAgain = smf->dIonizeTime+smf->dTime;!\\
\verb!      if (tCoolAgain > q->fTimeCoolIsOffUntil) q->fTimeCoolIsOffUntil=tCoolAgain;!\\
\verb!if (T < smf->dIonizeT) {!\\
\verb!	  CoolInitEnergyAndParticleData( smf->pkd->Cool, &q->CoolParticle, &q->u, q->fDensity, !\\
\verb!smf->dIonizeT, q->fMetals );!\\
\verb!	  }!\\\\
To this:\\\\
\verb!if (T < smf->dIonizeT) {!\\
\verb!      mgot += q->fMass;!\\
\verb!      ngot++;!\\
\verb!      tCoolAgain = smf->dIonizeTime+smf->dTime;!\\
\verb!      if (tCoolAgain > q->fTimeCoolIsOffUntil) q->fTimeCoolIsOffUntil=tCoolAgain;!\\
\verb!	  CoolInitEnergyAndParticleData( smf->pkd->Cool, &q->CoolParticle, &q->u, q->fDensity, !\\
\verb!smf->dIonizeT, q->fMetals );!\\
\verb!	  }!\\
\subsection*{Friday}
I've created a nice little script for plotting the difference between the 
ionization and no ionization version of Tom's ICs.  It works by doing a 2D
histogram of the $\rho-T$ diagram, and then just subtracting the bin values for
the two versions.  It's at 
\verb!Research/code/scripts/compare_T_rho_Tom_IC_DRTFORCE2.py!.  As it shows, 
there is some interesting stuff going on, where the ionization seems to be 
upping the density in the S-shaped part of the phase diagram, while also 
depleting the highest density chunk of the high-temperature string of particles.
\end{document}
