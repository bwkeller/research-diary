%%% Research Diary - Entry
\documentclass[11pt,letterpaper]{article}

% Working date: The date this entry is describing. Not necessary the date of last edit.
\newcommand{\workingDate}{\textsc{2012 $|$ July $|$ 16}}

% Name and institution must preceed call to researchdiary.sty package
\newcommand{\userName}{Ben Keller}
\newcommand{\institution}{McMaster University}

% The guts and glory
\usepackage{researchdiary}

% Begin document
\begin{document}

\subsection*{To Do}
\begin{bullets}
\item \checkmark Fix the Big version's initial star mass
\item \checkmark Re-run everything (turbulent) with $10km/s$ turbulence.
\item \textleaf Run with 4 glommed together versions of the IC (in the x-y plane)
\item Hunt down sources of periodicity in SFR
\item \checkmark Plot energy budget in Chris' IC
\end{bullets}

\textleaf : \textit{In Progress} \qquad \checkmark : \textit{Completed}

\section*{Daily Log}
\subsection*{Monday}
I met with James today, and we discussed the results of the $10 km/s$ turbulent
runs.  As the plots show, the crazy burst/collapse effect at the beginning was
from the turbulent pressure: it is totally eliminated in these runs.  James
suggest I hunt down the smaller amplituded/low $\mathbf{k}$ fluctuations:
are they bin result? Or are they something caused reflections from the box?
\subsection*{Tuesday}
Made some plots. Yup.
\subsection*{Wednesday}
I just had the most intense marathon meeting with James today.  210 minutes of
pure discussion.  We've come to the conclusion (based on a paper on "Diffuse 
Nebulae" from 1969) that HII regions are too big to be considered "subgrid"
problems, and we likely need real radiative transfer in order to actually 
model the physics there.

\subsection*{Thursday}
As a result of yesterday's discussion, I'm going to see what I can do to 
speed up gasoline.  I've started with some basic profiling with gprof. 
The test data is at \verb!Research/data/gasoline_performance!, and I've started
a performance branch of gasoline in git.  The test parameter runs in 317.26s
in vanilla gasoline.  The biggest share of the execution time is in 
\verb!pkdBucketInteract!, the gravity solver function. I'm going to see if 
\verb!#define NATIVE_SQRT! speeds things up.  Essentially no change.

Tomorrow I would like to meet with James to discuss the results of the wide
box runs.  I've restarted the last two with bigger CPU numbers, hopefully
that will speed things up.
%\subsection*{Friday}
%\subsection*{Saturday}
%\subsection*{Sunday}

\end{document}
