%%% Research Diary - Entry
\documentclass[11pt,letterpaper]{article}

% Working date: The date this entry is describing. Not necessary the date of last edit.
\newcommand{\workingDate}{\textsc{2011 $|$ Nov $|$ 21}}

% Name and institution must preceed call to researchdiary.sty package
\newcommand{\userName}{Ben Keller}
\newcommand{\institution}{McMaster University}

% The guts and glory
\usepackage{researchdiary}

% Begin document
\begin{document}

\subsection*{To Do}
\begin{bullets}
\item[\checkmark] Catch up on ridiculous arXiv backlog
\item[\checkmark] Run all the default examples in ChaNGa
\item[\textleaf] See where to start with gasoline
\item Familiarize myself with tipsy
\end{bullets}

\textleaf : \textit{In Progress} \qquad \checkmark : \textit{Completed}

\section*{Daily Log}
\subsection*{Thursday}
I was finally able to get ChaNGa to run Tom's ICs today.  I had to install the GNU Scientific Library (GSL), and do some modifications to the Makefile generated
by configure in order to get it to successfully use all of the configuration 
options in \verb!Disk_Collapse_1e6.param!.  I ran configure with the option
\verb!./configure --enable-cooling=cosmo! and added the following to the final
generated Makefile:\\
Added @ Line 11: \verb!BEN = -DDIFFUSION!\\
Appended to the end of Line 62: \verb!$(BEN)! (Line 62 reads \verb!DEFINE_FLAGS = ...!)\\
After compiling with these options (NOTE: \verb!make clean! can be ran to 
obliterate old compiled versions), Tom's ICs totally run.  I spent hours trying
to get ChaNGa to run on a SHARCNet machine (requin, saw, and orca) unsuccesfully
this evening, and decided to run the ICs on angband, which should complete the
run this weekend.  ChaNGa has a bunch of output, which I will need to analyze 
and document.  I will also need to more thouroughly attempt a SHARCNet run of
ChaNGa, and document what does and does not work.  I will need to use requin, as
it is the only machine that includes the latest (6.3.1) version of Charm++.

\subsection*{Friday}
I started to run ChaNGa on angband last night, and it is running smoothly this
morning.  A back of the envelope calculation estimates it should complete in the
morning tomorrow.  I will have to look into compiling it with CUDA support, and
see if that is very fast on angband.  It would please me greatly to be able to
compete with requin for speed with my own personal machine!\\
Today I am trying to get gasoline to run Tom's ICs.  I've downloaded all of the
\verb!pkdgrav! directory from \verb!imp:/home/wadsley/pkdgrav/!, so I will now
try and pick through it to find what is the most current version of gasoline.
\end{document}
